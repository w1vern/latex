\documentclass[a4paper, 12pt]{extarticle}

% Подключение пакетов для поддержки Unicode и русского языка
\usepackage{lmodern}        % Современные шрифты
\usepackage{amsmath}        % Математические формулы
\usepackage{graphicx}       % Работа с изображениями
\usepackage{hyperref}       % Гиперссылки
%\usepackage[table,xcdraw]{xcolor} % Цвета для таблиц
\usepackage{geometry}       % Настройка полей

\usepackage{fontspec}      % Работа со шрифтами
\usepackage{polyglossia}   % Поддержка многоязычия
\usepackage{anyfontsize}
\usepackage{pgfplots}
\usepackage{tikz}
\usepackage{xcolor}

\usepgflibrary {shadings}

%\pgfplotsset{compat=1.18}

% Установка языка
\setdefaultlanguage{russian}

% Установка шрифта с поддержкой кириллицы
\setmainfont{Times New Roman} % Или другой шрифт, поддерживающий кириллицу

\pagecolor{gray!176.75}
\color{white}



\title{Примеры документа в LaTeX}
\author{Автор: Богдан Дягилев}
\date{\today}

\begin{document}


\maketitle % Заголовок документа
\tableofcontents % Генерация оглавления
\newpage

\section{Введение}
LaTeX — это мощная система вёрстки. С её помощью можно создавать документы любого уровня сложности, включая книги, статьи и презентации.

\subsection{Особенности LaTeX}

Основные преимущества:
\begin{itemize}
    \item Высокое качество верстки текста.
    \item Поддержка сложных математических формул.
    \item Удобная работа с таблицами и изображениями.
    \item Автоматическая генерация оглавления, списков и ссылок.
\end{itemize}

\subsection{Пример кода}

Это пример использования полужирного шрифта (\textbf{жирный}), \textit{курсива} и \underline{подчёркивания}. Вот пример гиперссылки: \href{https://www.latex-project.org}{официальный сайт LaTeX}.

\section{Формулы}

Встроенная поддержка математики — одно из главных достоинств LaTeX. Вот несколько примеров:
\begin{itemize}
    \item Формула в строке: \(E = mc^2\).
    \item Формула отдельно:
          \[
              a^2 + b^2 = c^2
          \]
    \item Более сложная формула:
          \[
              \int_{0}^{\infty} e^{-x^2} \, dx = \frac{\sqrt{\pi}}{2}
          \]
\end{itemize}

\section{Списки}

\subsection{Нумерованный список}
\begin{enumerate}
    \item Первый элемент.
    \item Второй элемент.
    \item Третий элемент.
\end{enumerate}

\subsection{Маркированный список}
\begin{itemize}
    \item Элемент A.
    \item Элемент B.
\end{itemize}

\section{Таблицы}

Пример простой таблицы:

\begin{table}[h!]
    \begin{tabular}{l|r|c}
        \hline
        \textbf{Заголовок 1} & \textbf{Заголовок 2} & \textbf{Заголовок 3} \\ \hline
        Данные 1             & Данные 2             & Данные 3             \\ \hline
        10                   & 20                   & 30                   \\ \hline
    \end{tabular}
    \caption{Пример таблицы}
    \label{tab:example}
\end{table}

\section{Изображения}

Вставка изображения с подписью:
\begin{figure}[h!]
    \centering
    \includegraphics[width=0.5\textwidth]{example-image} % Замените example-image на имя файла
    \caption{Пример изображения}
    \label{fig:example}
\end{figure}

\section{Ссылки}

Внутренняя ссылка: см. Таблицу~\ref{tab:example}. \\
Внешняя ссылка: \href{https://www.latex-project.org}{LaTeX Project}.

\section{Пример графика}
\begin{tikzpicture}
    \begin{axis}[
            xlabel={$x$},         % Подпись оси X
            ylabel={$y$},         % Подпись оси Y
            title={Пример графика функции $y = x^2$}, % Заголовок графика
            grid=major,           % Отображение сетки
            legend style={at={(0.5,-0.2)},anchor=north,legend columns=-1, fill=black, text=white} % Настройка легенды
        ]
        % Добавляем функцию
        \addplot[
            domain=-2:2,      % Диапазон x
            samples=100,      % Количество точек
            color=blue,       % Цвет линии
            thick             % Толщина линии
        ]
        {x^2};
        \addlegendentry{$y = x^2$\newline}

        % Добавляем еще одну функцию
        \addplot[
            domain=-2:2,
            samples=100,
            color=red,
            % Пунктирная линия
        ]
        {x^3};
        \addlegendentry{$y = x^3$}
    \end{axis}
\end{tikzpicture}

\section{Пример градиента}


% Определение градиента
%\definecolor{startcolor}{RGB}{255, 0, 0} % Начальный цвет (красный оттенок)
%\definecolor{endcolor}{RGB}{0, 0, 255}   % Конечный цвет (синий оттенок)

% Рисуем прямоугольник с градиентом



\begin{tikzpicture}
    \definecolor{startcolor}{RGB}{255, 0, 0} % Начальный цвет (красный оттенок)
    \definecolor{endcolor}{RGB}{0, 255, 0}   % Конечный цвет (синий оттенок)
    \shade[shading=axis,
        left color=startcolor,
        middle color=white!50,
        right color=endcolor]
    (0, 0) rectangle (10, 5);
\end{tikzpicture}



\section{Заключение}

LaTeX — это универсальный инструмент для создания документов. Попробуйте его, чтобы раскрыть все возможности качественной верстки.

\end{document}
